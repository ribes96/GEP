\documentclass[a4paper]{article}
\usepackage[utf8]{inputenc}
\usepackage[english]{babel}
\usepackage{biblatex}
\usepackage{lmodern,textcomp}
\usepackage{caption}
\usepackage{hyperref}
\hypersetup{
    colorlinks,
%     citecolor=black,
    % filecolor=black
    linkcolor=blue,
%     urlcolor=black
}

\addbibresource{sample.bib}




\title{
Using Random Fourier Features with \\ Random Forests \\
\large Deliverable 3: Budget and Sustainability}
\author{Albert Ribes Marzá}

\begin{document}
    \maketitle
    \pagebreak

    \tableofcontents
    \pagebreak


    \section{Self-assessment on sustainability}
    % En 300 palabras, lo que he sacado de la encuesta

    % Poner todos los puntos fuertes y débiles


%     De cara a realizar un proyecto de ingeniería, las actitudes de entrada que tengo son muy buenas. Tengo muy clara la definición de sostenibilidad, conozco sus tres dimensiones, y soy consciente de que un proyecto debe intentar equilibrar las tres en la medida de lo posible.
%

When carrying out an engineering project, the first attitude is the best. I know very well the definition of sustainability, its three dimensions, and I am conscious that a project should try to balance the three to the extent possible.

I am aware of the damage that can be done to the planet with a decision in a project, and I appreciate a project for being respectful with the environment, even if I don't notice directly the benefits. I also have a good image of the current situation in this aspect in the world.

Socially, I care about social justice, equity, diversity and transparency. I know that a project only is good if it contributes to the benefit of the citizens, and I know the social consequences that a project I develop could have.

I consider that the purpose of a project should never be to earn money, but see it as a mean, and something needed to get it off the ground.

On the other hand, I don't have the means to get real information and data regarding the environment. I neither know how to evaluate the recycling process a product will have after it useful life, and I am not able to do correct calculations about the pollution of a product in the environment.

I also can't know the social consequences that mi products are having, nor if they are really being used for what I expected. Regarding colaborarive work, I know the tools and techniques to carry projects out, but in practice I don't know how to use it well, or I don't make use of it in the best way.

With the economical dimension it happens a similar thing. I know the theory, but in practice I don't know the means to recognize if a project is producing the expected goods.

% Soy consciente del daño que se le puede hacer al planeta con una decisión en un proyecto, y valoro que un proyecto sea respetuoso con el medio ambiente, incluso si los beneficios de este aspecto no los percibo yo directamente. También tengo una idea aproximada de cual es la situación actual en este aspecto en el mundo.
%
% Socialmente, me preocupa la justicia social, la equidad, la diversidad y la transparencia. Sé que un proyecto solamente es bueno si contribuye al bien de los ciudadados, y sé qué consecuencias sociales pueden tener los proyectos que desarroye.
%
% Considero que el fin de un proyecto nunca puede ser conseguir dinero, sino que éste debe ser un medio, y algo necesario para sacarlo adelante.
%
% Por otro lado, no dispongo de los medios para conseguir información y datos reales referentes al medio ambiente. Tampoco sé cómo evaluar el proceso de reciclaje que tendrá un producto después de su vida útil, y no soy capaz de hacer cálculos correctos sobre la contaminación que puede tener un producto en el medio ambiente.
%
% Tampoco tengo los medios para saber las consecuencias sociales que están teniendo mis productos, ni si realmente se están usando para lo que yo esperaba. Respecto al trabajo colaborativo, conozco las herramientas y técnicas para llevarlo a cabo, pero en la práctica no sé utlizarlas bien, o no las exploto al máximo.
%
% Con la dimensión económica ocurre algo parecido. Conozco la teoría, pero en la práctica no conozco los medios para saber si un proyecto está produciendo los bienes que estaban esperados.




    \section{Analysis of the sustainability of the project}
        \subsection{Environmental dimension}
            \subsubsection{PPP}

            During the production process of the project there will be needed power supply for charging the computer and general business resources such as light, heating, etc.

            The computer consumption is 10 W, and although it is not always the same, I will approximate the business resources as being 0.24kW, so the total consumption is 0.25 kW. As the project is planned to last 240 hours, the amount of power spent will be \(0.25 kW \times 240 h = 60 kWh\).

            The project doesn't generate residues.

            % El impacto sobre el medio ambiente a lo largo de la realización del TFG (consumo energético y generación de residuos).
            %
            % El único consumo que hay es el uso del ordenador para todo.
            %
            % No se genera ningún residuo.
            %
            % He considerado que el proyecto durará 240 horas. Durante todas esas horas el ordenador estará encendido. El consumo que tiene un ordenador asumo que es 0’125 kWh. Hacer el cálculo.
            %
            % Mi ordenador dice que el rate de consumo es de 9.5 W, esto es, cada segundo gasta 9.5 Joules. Puesto que el trabajo tiene que durar 240 horas, calculo que el total del trabajo va a consumir unos 22.8 kWh (esto es una unidad de trabajo, podría haber usado los Joules, pero habrían sido muchos. La potencia que gasta el ordenador es de 9.5 W)
            \subsubsection{Useful life}

            As this is project is a very theoretical study, it is hard to say what will be the environmental impact it will have during its useful life. If the conclusion of the project is satisfactory, it may help to develop more efficient machine learning algorithms which will need less energy spends and thus decrease the energy consumption.

            % La huella ecológica que tendrá el proyecto durante toda su vida útil
            %
            % Es un simple trabajo de investigación, no creo que tenga ninguna huella ecológica.
            %
            % Si sale un algoritmo más eficiente, puede repercutir en el consumo que hagan las máquinas de otros para hacer machine learning.

            \subsubsection{Risks}

            The worst it could happen is that the conclusions reached with the project are useless and they do not help to improve efficiency in the machine learning algorithms.

            % El conjunto de eventualidades que podrían causar que el impacto ambiental del proyecto sea más negativo del previsto
            %
            % Realmente no se me ocurre nada
            %
            % Hemos previsto que quizá hace que los algoritmos consuman más. Lo negativo que puede pasar es que no mejore nada.



        \subsection{Economic dimension}
            \subsubsection{Budget}

            \begin{table}
                \centering
                \begin{tabular}{|c|c|c|c|}
                    \hline
                    \textbf{Task} & \textbf{Time (h)} & \textbf{Money Spent (€)} \\
                    \hline
                    Backgroung Approximation & 50 & 1500 \\
                    \hline
                    Decide the kernel & 6 & 180 \\
                    \hline
                    Decide Dimensionality & 4 & 120 \\
                    \hline
                    Decide the changes & 10 & 300 \\
                    \hline
                    Implement Fourier mapping & 10 & 300 \\
                    \hline
                    Get familiar with the module & 20 & 600 \\
                    \hline
                    Modify the module & 20 & 600 \\
                    \hline
                    Debug the code & 15 & 450 \\
                    \hline
                    Find testing datasets & 5 & 150 \\
                    \hline
                    Accuracy tests & 10 & 300 \\
                    \hline
                    Time tests & 10 & 300 \\
                    \hline
                    Study the results & 10 & 300 \\
                    \hline
                    Repeat parts after testing & 20 & 600 \\
                    \hline
                    Composition of the document & 30 & 900 \\
                    \hline
                    \textbf{Total} & \textbf{240} & \textbf{7200} \\
                    \hline

                \end{tabular}
                \caption{It is considered a salary of 30 € / hour}
                \label{Tab:1}
            \end{table}


            \subsubsection*{Direct costs}

            All the software used in this project will be free, and thus they don't increase the cost of the project.

            The cost of the laptop will be specified in the ``Depreciation'' section.

            The workforce of the project consists of a single person working for 240 hours. Assuming a salary of 30 €/hour, the labour costs will be 240 hours \(\times\) 30€/hour = 7200€.

            Table \ref{Tab:1} shows the direct costs for each task in the project, and the costs of the whole project are summarized in table \ref{Tab:2}

            \subsubsection*{Indirect costs}

            As the workspace of the project will be the facilities of the FIB, transport service will be needed. I will use the public transport, so the cost will be 150 €.

            \subsubsection*{Depreciation}

            The cost of the computer needs to be depreciated. I expect it to have a lifetime of 7500 hours, and the project will spend 240, so the cost is a $3.2 \%$ of the total price of the computer. As it was 800 €, the cost of the project is $3.2 \% \times 800 = 25.6$ €.

            % Hay que tener los campos:
            % \begin{itemize}
            %     \item Costes directos por actividad
            %
            %         Indicar que todo el software que uso es libre, y por lo tanto no tengo costes de este estilo.
            %
            %         A pesar de que no tenga costes, quizá conviene hacer una tabla, y que para cada una ponga que tiene coste = 0
            %
            %         El coste que ha tenido mi ordenador también entra aquí
            %
            %         Al final es coste humano y coste material, y el software está dentro de material
            %
            %         El coste del ordenador es de 800 €
            %
            %         Espero que el ordenador dure 4 años
            %
            %         Mirar el coste que tiene tenerme a mí trabajando esa cantidad de horas.
            %
            %
            %         El consumo de energía que ha tenido mi ordenador mientras hacía la investigación. Quizá también la luz que he gastado mientras hacía el proyecto.
            %
            %         El acceso a internet que he tenido que tener para hacer el proyecto.
            %
            %         \paragraph{Fixed}
            %
            %         \begin{itemize}
            %             \item La conexión a internet
            %             \item La luz que consumo
            %         \end{itemize}
            %
            %         \paragraph{Variables}
            %
            %         Creo que no hay




            % \begin{table}
            %     \centering
            %     \label{costes}
            %     \begin{tabular}{|c|c|c|c|}
            %         \hline
            %         \textbf{Task} & \textbf{Time (h)} & \textbf{Money Spent (€)} \\
            %         \hline
            %         Backgroung Approximation & 50 & 1500 \\
            %         \hline
            %         Decide the kernel & 6 & 180 \\
            %         \hline
            %         Decide Dimensionality & 4 & 120 \\
            %         \hline
            %         Decide the changes & 10 & 300 \\
            %         \hline
            %         Implement Fourier mapping & 10 & 300 \\
            %         \hline
            %         Get familiar with the module & 20 & 600 \\
            %         \hline
            %         Modify the module & 20 & 600 \\
            %         \hline
            %         Debug the code & 15 & 450 \\
            %         \hline
            %         Find testing datasets & 5 & 150 \\
            %         \hline
            %         Accuracy tests & 10 & 300 \\
            %         \hline
            %         Time tests & 10 & 300 \\
            %         \hline
            %         Study the results & 10 & 300 \\
            %         \hline
            %         Repeat parts after testing & 20 & 600 \\
            %         \hline
            %         Composition of the document & 30 & 900 \\
            %         \hline
            %         \textbf{Total} & \textbf{240} & \textbf{7200} \\
            %         \hline
            %
            %     \end{tabular}
            %     \caption{It is considered a salary of 30 € / hour}
            % \end{table}






                    % \begin{table}
                    %     \centering
                    %     \begin{tabular}{|c|c|c|c|}
                    %         \hline
                    %         \textbf{Task} & \textbf{Time (h)} & \textbf{Power Consumption (Wh)} & \textbf{Money Spent (cnts)} \\
                    %         \hline
                    %         Backgroung Approximation & 50 & 500 & 0.5 \\
                    %         \hline
                    %         Decide the kernel & 6 & 60 & 0.06 \\
                    %         \hline
                    %         Decide Dimensionality & 4 & 40 & 0.04 \\
                    %         \hline
                    %         Decide the changes & 10 & 100 & 0.1 \\
                    %         \hline
                    %         Implement Fourier mapping & 10 & 100 & 0.1 \\
                    %         \hline
                    %         Get familiar with the module & 20 & 200 & 0.2 \\
                    %         \hline
                    %         Modify the module & 20 & 200 & 0.2 \\
                    %         \hline
                    %         Debug the code & 15 & 150 & 0.15 \\
                    %         \hline
                    %         Find testing datasets & 5 & 50 & 0.05 \\
                    %         \hline
                    %         Accuracy tests & 10 & 100 & 0.1 \\
                    %         \hline
                    %         Time tests & 10 & 100 & 0.1 \\
                    %         \hline
                    %         Study the results & 10 & 100 & 0.1 \\
                    %         \hline
                    %         Repeat parts after testing & 20 & 200 & 0.2 \\
                    %         \hline
                    %         Composition of the document & 30 & 300 & 0.3 \\
                    %         \hline
                    %         \textbf{Total} & \textbf{240} & \textbf{2400} & \textbf{2.4} \\
                    %         \hline
                    %
                    %     \end{tabular}
                    %     \caption{Consideraremos que el precio de la energía es 0.10 € / KWh}
                    % \end{table}

                % \item Costes indirectos
                %
                % El transporte que necesito hacer hasta la universidad
                %
                %
                % \item Amortizaciones
                %
                % Espero que el ordenador me dure 4 años
                %
                % El ordendador me ha costado 800 €.
                %
                % Al proyecto le voy a dedicar 240 horas de trabajo.
                %
                % Calculo que me tiene que durar 7500 horas de trabajo.
                %
                % La proporción que gasta el proyecto es 240 / 7500 = 3.2 \%
                %
                % Por lo tanto, lo que gasta el proyecto del portátil es 3.2 * 800 / 100 = 25.6 €

                % \item Contingencias e imprevistos

                % Quizá se me estropea el ordenador
            % \end{itemize}

            \subsubsection*{Unforeseen contingencies}

            The computer I plan to use to develop the project could have a breakdown. If this happens, the reparation or even replacement will increase the cost of the project.

            In order to avoid losing the work done, all the data will be securely stored in GitHub, where it is very unlikely to be lost.

            % Quizá se me estropea el ordenador
            % La estimación de costes se hace a nivel de actividades del Gantt, que son:
            % \begin{itemize}
            %     \item Background approximation
            %     \item Decide the kernel
            %     \item Decide dimensionality
            %     \item Decide the changes
            %     \item Implement Fourier Mapping
            %     \item Get familiar with the module
            %     \item Modify the module
            %     \item Debug de code
            %     \item Find testing dataset
            %     \item Preprocessing
            %     \item Accuracy tests
            %     \item Time tests
            %     \item Study the results
            %
            % \end{itemize}
            %
            % Hay que estimar a cuanto serán esos costes y los mecanismos que tendré para evitar desviaciones
            %
            %
            % Para cada uno, recorrerme la lista de tareas que tengo hecha del deliverable anterior, y ver si tienen algo de esto. Poner la estimación de lo que creo que será
            %
            % \subsubsection*{Direct costs}
            % Las horas de trabajo, que se copian de la entrega anterior, y el consumo de energía del ordenador.
            %
            % El consumo debe ser igual al de la PS4, que es 0’125 kWh cada hora

            % \subsubsection*{Indirect Costs}
            % \subsubsection*{Depreciation}


            \begin{table}
                \centering
                % \label{}
                \begin{tabular}{|c|c|c|}
                    \hline
                    \textbf{Subject} & \textbf{Type} & \textbf{Amount (€)} \\
                    \hline
                    \hline
                    Workforce & Direct & 7200 \\
                    \hline
                    Transport & Indirect & 150 \\
                    \hline
                    Power & Fixed & Undefined$^*$ \\
                    \hline
                    Computer & Depreciation & 25.6 \\
                    \hline
                    \textbf{Total} & & 7375.6 \\
                    \hline
                \end{tabular}
                \caption{Total costs of the project}
                \label{Tab:2}
                $^*$ It is not possible to calculate that quantity since I will be using the facilities of the FIB
            \end{table}

            \subsubsection{Assessment}


            I expected that the costs for a project as ``simple'' as this one would be much fewer, since it involves very little extra costs apart from the salary of the worker. I find the cost is very high given the case that good results are not guaranteed, and I've learned that it is very important to first study the costs of a project, since the conclusions could not be intuitive.

            % \begin{itemize}
            %     \item Reflexionar sobre el coste que he estimado para la realización
            %     \item ¿Cómo se resuelven actualmente los problemas que quiere resolver mi proyecto?
            %     \item En qué mejora económicamente mi solución respecto de las otras existentes (en costes)
            % \end{itemize}



        \subsection{Social dimension}
            \subsubsection{PPP}

            This project will teach me how to correctly plan and develop a good working methodology. As this is the fists big project I have to do, it will be very useful to see if only having good programming skills is enough to make projects get on. It will also teach me the easiest failure points in a big project, and will allow me to avoid them in more important projects in the future.
            %
            % El impacto que ha tenido la realización del proyecto sobre las personas que han trabajado en él. Reflexionar sobre los cambios que la realización del proyecto ha provocado en mí y en mi entorno más directo.
            %
            % A mí me está ayudando a aprender a hacer proyectos propiamente dichos, a planificarlos bien y todo eso. He tenido que organizarme mejor y todo eso.

            \subsubsection{Useful life}


            Right now the algorithms used for training the Random Forest use some special kind of bagging which allows each node of the trees to use just a partition of the whole data. It is possible in this project that we find a better way to do the bagging allowing each node to see the entire dataset without loosing performance, or even improving it.

            It is possible that this project reaches satisfactory results, and it is possible that the method is proven not to be useful. In the first case it will help future studies to train a better learner, and in the second one, it will make other researchers know that this is not a good method, so they will not need to invest resources in performing the same study.

            Better learning algorithms are needed nowadays, so there exists a real need of this project and many others to try to develop them.

            % El impacto que tendrá la puesta en marcha del proyecto sobre los colectivos relacionados, ya sea de forma directa o indirecta.
            %
            % Quizá puedo separar en dos secciones, en el caso que las conclusiones sean satisfactorias y en el caso de que no lo sean.
            %
            % En el caso que este proyecto llegue a resultados satisfactorios, la comunidad científica dispondrá de una nueva técnica de machine learning con buenos resultados. También es posible que gracias a la realización de este proyecto, se puedan realizar otros productos que permitan hacer inferencia gastando menos recursos de energía, espacio de almacenamiento y tiempo de cálculo.
            %
            % Si los resultados no son satisfactorios, es decir, las modificaciones que hacemos no parecen mejorar el rendimiento de los random forest, este es un estudio que lo comprueba, y por tanto evitará que otras personas vuelvan a hacer el mismo estudio, lo cual sería una pérdida de tiempo.

            \subsubsection{Risks}

            As the aim of this study is to provide knowledge, there are not real risks for this project. The worst it could happen is that the conclusions reached with the project are useless.

            %
            % Eventualidades que que podrían causar que el impacto social que el proyecto sobre alguno de los colectivos relacionados con él sea más negativo del previsto
            %
            %
            % Realmente creo que no existe ningún riesgo con este proyecto. Tan solo aporta conocimiento al asunto.




    %
    % \section{Project Budget}
    % \section{Budget Monitoring}
    % \section{Sustainability and social commitment}








\printbibliography


\end{document}
