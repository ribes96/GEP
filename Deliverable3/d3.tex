\documentclass[a4paper]{article}
\usepackage[utf8]{inputenc}
\usepackage[english]{babel}
\usepackage{biblatex}
\usepackage{hyperref}
\hypersetup{
    colorlinks,
%     citecolor=black,
    % filecolor=black
    linkcolor=blue,
%     urlcolor=black
}

\addbibresource{sample.bib}




\title{
Using Random Fourier Features with \\ Random Forests \\
\large Deliverable 3: Budget and Sustainability}
\author{Albert Ribes Marzá}

\begin{document}
    \maketitle
    \pagebreak

    \tableofcontents
    \pagebreak


    \section{Self-assessment on sustainability}
    En 300 palabras, lo que he sacado de la encuesta

    Poner todos los puntos fuertes y débiles


patata patata patata patata patata patata patata patata patata patata patata
patata patata patata patata patata patata patata patata patata patata patata
patata patata patata patata patata patata patata patata patata patata patata
patata patata patata patata patata patata patata patata patata patata patata
patata patata patata patata patata patata patata patata patata patata patata
patata patata patata patata patata patata patata patata patata patata patata
patata patata patata patata patata patata patata patata patata patata patata
patata patata patata patata patata patata patata patata patata patata patata
patata patata patata patata patata patata patata patata patata patata patata
patata patata patata patata patata patata patata patata patata patata patata
patata patata patata patata patata patata patata patata patata patata patata
patata patata patata patata patata patata patata patata patata patata patata
patata patata patata patata patata patata patata patata patata patata patata
patata patata patata patata patata patata patata patata patata patata patata
patata patata patata patata patata patata patata patata patata patata patata
patata patata patata patata patata patata patata patata patata patata patata
patata patata patata patata patata patata patata patata patata patata patata
patata patata patata patata patata patata patata patata patata patata patata
patata patata patata patata patata patata patata patata patata patata patata
patata patata patata patata patata patata patata patata patata patata patata
patata patata patata patata patata patata patata patata patata patata patata
patata patata patata patata patata patata patata patata patata patata patata
patata patata patata patata patata patata patata patata patata patata patata
patata patata patata patata patata patata patata patata patata patata patata
patata patata patata patata patata patata patata patata patata patata patata
patata patata patata patata patata patata patata patata patata patata patata
patata patata patata patata patata patata patata patata patata patata patata

    \section{Analysis of the sustainability of the project}
        \subsection{Environmental dimension}
            \subsubsection{PPP}
            El impacto sobre el medio ambiente a lo largo de la realización del TFG (consumo energético y generación de residuos).

            El único consumo que hay es el uso del ordenador para todo.

            No se genera ningún residuo.
            \subsubsection{Useful life}

            La huella ecológica que tendrá el proyecto durante toda su vida útil

            Es un simple trabajo de investigación, no creo que tenga ninguna huella ecológica.

            Si sale un algoritmo más eficiente, puede repercutir en el consumo que hagan las máquinas de otros para hacer machine learning.

            \subsubsection{Risks}

            El conjunto de eventualidades que podrían causar que el impacto ambiental del proyecto sea más negativo del previsto

            Realmente no se me ocurre nada



        \subsection{Economic dimension}
            \subsubsection{Budget}
            Hay que tener los campos:
            \begin{itemize}
                \item Costes directos por actividad

                    Indicar que todo el software que uso es libre, y por lo tanto no tengo costes de este estilo.

                    A pesar de que no tenga costes, quizá conviene hacer una tabla, y que para cada una ponga que tiene coste = 0

                    El coste que ha tenido mi ordenador también entra aquí

                    Al final es coste humano y coste material, y el software está dentro de material

                \item Costes indirectos

                    El consumo de energía que ha tenido mi ordenador mientras hacía la investigación. Quizá también la luz que he gastado mientras hacía el proyecto.

                    El acceso a internet que he tenido que tener para hacer el proyecto.

                \item Amortizaciones

                Espero que el ordenador me dure 4 años

                \item Contingencias e imprevistos

                Quizá se me estropea el ordenador
            \end{itemize}
            La estimación de costes se hace a nivel de actividades del Gantt, que son:
            \begin{itemize}
                \item Background approximation
                \item Decide the kernel
                \item Decide dimensionality
                \item Decide the changes
                \item Implement Fourier Mapping
                \item Get familiar with the module
                \item Modify the module
                \item Debug de code
                \item Find testing dataset
                \item Preprocessing
                \item Accuracy tests
                \item Time tests
                \item Study the results

            \end{itemize}

            Hay que estimar a cuanto serán esos costes y los mecanismos que tendré para evitar desviaciones


            Para cada uno, recorrerme la lista de tareas que tengo hecha del deliverable anterior, y ver si tienen algo de esto. Poner la estimación de lo que creo que será

            \subsubsection*{Direct costs}
            Las horas de trabajo, que se copian de la entrega anterior, y el consumo de energía del ordenador.

            El consumo debe ser igual al de la PS4, que es 0’125 kWh cada hora

            \subsubsection*{Indirect Costs}
            \subsubsection*{Depreciation}
            \subsubsection*{Unforeseen contingencies}

            \subsubsection{Assessment}

            \begin{itemize}
                \item Reflexionar sobre el coste que he estimado para la realización
                \item ¿Cómo se resuelven actualmente los problemas que quiere resolver mi proyecto?
                \item En qué mejora económicamente mi solución respecto de las otras existentes (en costes)
            \end{itemize}



        \subsection{Social dimension}
            \subsubsection{PPP}

            El impacto que ha tenido la realización del proyecto sobre las personas que han trabajado en él. Reflexionar sobre los cambios que la realización del proyecto ha provocado en mí y en mi entorno más directo.

            A mí me está ayudando a aprender a hacer proyectos propiamente dichos, a planificarlos bien y todo eso. He tenido que organizarme mejor y todo eso.

            \subsubsection{Useful life}

            El impacto que tendrá la puesta en marcha del proyecto sobre los colectivos relacionados, ya sea de forma directa o indirecta.

            Quizá puedo separar en dos secciones, en el caso que las conclusiones sean satisfactorias y en el caso de que no lo sean.

            En el caso que este proyecto llegue a resultados satisfactorios, la comunidad científica dispondrá de una nueva técnica de machine learning con buenos resultados. También es posible que gracias a la realización de este proyecto, se puedan realizar otros productos que permitan hacer inferencia gastando menos recursos de energía, espacio de almacenamiento y tiempo de cálculo.

            Si los resultados no son satisfactorios, es decir, las modificaciones que hacemos no parecen mejorar el rendimiento de los random forest, este es un estudio que lo comprueba, y por tanto evitará que otras personas vuelvan a hacer el mismo estudio, lo cual sería una pérdida de tiempo.

            \subsubsection{Risks}

            Eventualidades que que podrían causar que el impacto social que el proyecto sobre alguno de los colectivos relacionados con él sea más negativo del previsto


            Realmente creo que no existe ningún riesgo con este proyecto. Tan solo aporta conocimiento al asunto.




    % 
    % \section{Project Budget}
    % \section{Budget Monitoring}
    % \section{Sustainability and social commitment}








\printbibliography


\end{document}
